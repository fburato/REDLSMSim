\section{Introduction}
The current document describes the main aspects of the application developed
to simulate clone detection protocols in a Wireless Sensor Network (WSN).

A Wireless Sensor Network (WSN) consists of spatially distributed autonomous 
sensors to monitor physical or environmental conditions, such as temperature, 
sound, vibration, pressure, motion or pollutants and to cooperatively pass 
their data through the network to a main location. 
Today WSNs are usable in very different environments and for different purposes
but they were initially developed for military applications such as battlefield
surveillance. It should be evident that in such a contest security is very important because any interference caused by enemies could cause harm to soldiers and 
damage to buildings.

One of the possible attacks which can occur in a WSN is a ``clone attack''. 
Since sensors cannot be continuously monitored by people they can be captured 
by enemies, modified and replicated in multiple copies. In this scenario
every data collected in the network would be completely useless and unreliable 
because it could have been maliciously altered by cloned sensors.

Two possible protocol usable to detect a clone attack and simulated in this 
project are:
\begin{itemize}
  \item LSM: Line-Selected Multicast protocol.
  \item RED: Randomized, Efficient, and Distributed protocol.
\end{itemize}
