\section{Main working of the system}
The GUI of the program is made by a main frame (MainWindow) that contains five
panels used to configure and view the simulation. Following a description of 
each of these panels.

\begin{description}
  \item[GraphPanel.] This panel shows to the user the state of the simulation.
  Every node is drawn as a filled circle. At first, the color of each node is
  black except for the two clones generated for simulating the clone attack
  that are painted in blue. If arches are visible (see ButtonPanel), they are
  also painted in black. The destination generated in both of the two protocol 
  is drawn with a red cross. 
  When the claim message is sent, arches that link clones with their neighbors
  are painted in green. From the location claim on every arch that represents
  a successful send (not a successful receive) is coloured in green and every
  node that is forwarding a location claim coming from a clone and every 
  neighbour of clones that decides to forward the claim with probability `p' is
  coloured in green. When the clone attack is detected, the node that made
  the detection is repainted in red.
  \item[ButtonPanel.] This panel allows to set some parameters and provides some
  buttons to start, stop, pause a single test or pause the whole simulation.
  These commands will be disabled during the simulation. To enable them the
  user must pause the simulation or the test currently running. This is done
  because, given to the amount of threads that could be necessary to run certain
  simulations (1000 nodes implies at least 1000 different threads), the user
  interface cannot be very reactive. The only way to make it not completely
  insufferable is to put the threads running the simulation in not-runnable state
  and to make the needed changes.
  Following a brief description of the available parameters and buttons.
  \begin{itemize}
    \item \textit{Sleep Time.} Sets the amount of time in milliseconds that each 
    StationNode has to wait before it forwards a message. By default it is set to
    100 milliseconds but it is possible to set it in the range from 0 (no pause)
    to 1000 milliseconds with increments of 100 milliseconds.
    \item \textit{Node Size.} Sets the diameter of a node. By defaults it is set to 5
    pixels but it is possible to set it in the range from 1 to 50 pixels with
    increments of 1 pixel.
    \item \textit{Cross Size.} Sets the dimension of the destination cross. By default
    it is set to 10 pixels but it is possible to set it in the range from 5 to
    50 pixels.
    \item \textit{View Arches.} When enabled it forces GraphPanel to draw not 
    only the green arches representing a successful send but also every other 
    arches in the graph coloured in black. 
    \item \textit{Comfirm Buttons.} These buttons are used to confirm a change in their
    respective parameters. That means that when a user want to change a parameter
    during a paused simulation he has to push, before unpause the simulation, the
    confirm button for each parameter that had been modified in order to make 
    changes take effect. 
    \item \textit{Start.} This button simply start the simulation. It is enabled
    as long as long the simulation starts; after that it is disabled until the
    simulation naturally ends or it is stopped.
    \item \textit{Stop.} This button completely stops the simulation. It is enabled only when
    the simulation is started or when a test or the simulation is paused.
    \item \textit{Pause Test.} This button pause the current executing test. It is enabled
    only when the simulation is already started. The user has just to push this button
    again to unpause the test and continue the simulation.
    \item \textit{Pause Sim.} This button pause the simulation running. It is enabled only
    when the simulation is already started. The simulation will pause only when the
    current test executing is finished. The user just has to push this button again
    to unpause the simulation and continue it.
  \end{itemize}
  \item[SimulationState.] This panel shows to the user the actual state of the
  simulation, the protocol and the number of the test running.
  \item[ResultsPanel.] This panel is used to show to the user results of tests.
  Each test result is printed when the test is finished. It occupies a line and
  is printed in the same order as results are send to and printed by the RMI server
  that is following the given specifications. In order: the name of the protocol,
  the number of tests, the number of nodes, the communication radius of a StationNode,
  the probability for a neighbour StationNode to  process a claim message, the number
  of destination locations, the initial amount of energy for each StationNode, the
  energy spent to send a message, the energy spent to receive a message, the energy
  spent to check a signature, the minimum, maximum, average and standard
  deviation of the number of sent messages,  the number of received messages, 
  the number of signature checks, the amount of consumed energy, the number of 
  stored messages and finally if there was a clone detection (true or false).
  To divide a simulation result from another, a line of asterisks is printed at the
  end of the simulation. The user can't write in this area.
  \item[TextPanel.] This panel lets the user insert the path of the configuration
  file and the address of the RMI server. It is possible to insert a local path, if
  the file is saved into the user's file system, or a remote path, if the file is
  saved on a web site. The user just need to write the path into the `Parser Path'
  field and then specify the location through radio buttons. By default it is set
  to the PCD course site's configuration file.
  To set the RMI address the user has to write its address into the `Server Address'
  field. By default it is set to `localhost'. For flexibility, it is given the
  possibility to avoid the RMI server loading by checking the `Ignore' check box.
  When the user pushes button `Start', the application extracts information from
  the text areas, loads the configuration file and tries to connects to the 
  RMI server (or not) following the indications given from users in this panel.
\end{description}

It's worth to say that when any errors occurs (badly formatted configuration
file, impossibility to connect to the RMI server, etc...) the GUI informs the
users with a message dialog which reports the encountered error.
\\[1cm]
The RMI server is a command line tool that accepts just a String representing 
the path to the file which will be used to write the statistical data collected
from clients and communicates to the user if unexpected situation happened 
(i.e. the user has not enough privileges to write in the given file, there is 
an existing instance of a server running on the host, rmiregistry is not 
running, etc...). As standard behaviour if the file
specified already exists when the server is started it is overrided. By the
way the new content arriving from clients will be continuously appended as 
long as the server stays open.
