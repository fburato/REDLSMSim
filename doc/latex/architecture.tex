\section{Architecture Description}
The entire application object of this report was developed having as aim
the maximum possible extensibility of the application itself. The general
architecture reflects this intention at all.

In order to represent as much as possible the reality of the scenario, the
developers decided to divide the logical structure of the software in four
separated levels:
\begin{description}
  \item[Hardware:] represents the WSN itself and it is constituted of a set
  of sensors.
  \item[Software:] represents the protocol running on each sensor.
  \item[GUI:] it's the top layer which interact with the user so it is totally
  independent from the rest of the architecture.
  \item[Server:] it's a separated layer dedicated to RMI communication.
\end{description}

\subsection{Hardware}
While developing the `hardware' part it was clear that the most general object
which must be handled while modelling a network is a graph. This simple
assumption determined the necessity to describe a sensor as a node in a graph
capable of different actions.
Every class and interface used to model a sensor can be find in package `graph'
and its subpackages.
Following a short description of the most important interfaces.
\begin{description}
  \item[Node.] It's the most general object that could be imagined being
  part of a graph. It is a node in the mathematical sense with its 
  neighbours. Since the application should be extensible the arch which connect
  every node with its adjacent-set is weighed and the graph is oriented.
  \item[ColoredNode.] In the first draft of the project specification
  was asked to evaluate the graph connectivity and the most simple
  solution suitable for counting connected components and determine whether or 
  not a graph is connected was considered to be a graph colouring algorithm. 
  Specification changed during development so this feature is no more needed. 
  By the way this aspect was maintained for completeness.
  \item[MetricNode, EuclideanNode.] In a WSN each sensor has a position
  in a reference system. Since the position in an euclidean plane is just a 
  specification of a metric space, it was natural to create a MetricNode which
  has to be considered as a Node able to calculate a distance between it and 
  another position. An EuclideanNode is simply a specification of a MetricNode
  which has its position defined as a couple of coordinates in a Cartesian
  coordinate system. By the way every metric space in which a distance concept
  can be modelled (even toroidal spaces) is easy to implement from MetricNode.
  \item [CommunicationNode.] The most simple way to allow to objects
  entities to communicate is: give a unique address to each entity and give 
  them a mailbox in which messages could be put. A CommunicationNode is a Node
  that offers a method (`putMessage') used by other entities to put a message in its
  mailbox. The addressing system is simulated via the references of the nodes
  saved in the neighbours data structure. This implies that in a graph of
  CommunicationNodes each node could talk just to its neighbours (because
  adding unique addresses to each node would have implied the implementation
  of some sort of routing algorithm which was considered too hard to handle).
  \item [OperationalNode.] Each sensor is capable of performing 
  operations which consume energy. This behaviour is granted via 
  OperationalNode interface which must be considered a Node able to perform
  (`perform' method) an Operation with a given cost. `perform' method should communicate
  whether or not the OperationaNode had enough battery to perform the requested
  Operation.
  \item [BaseNode.] It is a non-oriented non-weighed graph's node.
  \item [StationNode.] It is the very only class used in the entire object
  as `hardware'. Essentially it include every aspect of a sensor:
  it is node in a non-oriented and non-weighed graph (is-a BaseNode), 
  it has a color (is-a ColoredNode), it has a position in a 
  Cartesian coordinate system (is-a EuclideanNode), it's able to communicate
  (is-a CommunicationNode) and is able to perform operations which consume 
  energy (is-a OperationalNode). It also has some interesting ad-hoc features
  used by `software' in the application like two mailboxes, one for stored 
  messages and one for messages which aren't yet computed by the protocol.
\end{description}

\subsection{Software}
For `software' we intend the set of threads necessary to execute the protocols.
Essentially it is imaginable as a program which is running using StationNode
as its computer. The packages containing the classes of the software part are
`threads' and `client'. The following are the most important classes to 
describe.
\begin{description}
  \item[Hypervisor.] It is the `operative system' of the project. It is a 
  thread which manages every thread created for the simulation and checks 
  if a clone has been found or the simulation has stopped because there were
  no more messages to process in StationNodes. In order to allow the Hypervisor
  to check a consistent state of the simulation, it must be ensured that when
  the Hypervisor asks to execute no thread of the simulation should execute 
  (because if the Hypervisor is checking the number of messages in a 
  StationNode and there are StationNodes executing, they could change the number
  of messages present in mailboxes).
  \item[ProtocolThread.] It is actually the real thread that executes the
  protocol on a StationNode. The `run' method is constituted of two parts: the
  common synchronization mechanism and the operative part. The common 
  synchronization mechanism consists in a set of instruction that allow the 
  Hypervisor to be informed that the ProtocolThread has stopped due to the 
  request of the Hypervisor itself. The operative part is designed as an abstract 
  method `protocol' which must be overrided by subclasses.
  \item[REDLSMCommon, RedProtocol, LSMProtocol.] These are the subclasses of
  ProtocolThread which executes the operative part of their superclass.
  
  REDLSMCommon is the common set of operations that both RED and LSM protocol
  executes. They both cause every node to claim their location to their
  neighbours and execute different operations when they process a claim message
  or a claim forwarded. The only difference between them is what they do when
  processing the messages so methods `processClaim' and `processForward' are
  abstract and must be overrided.
  
  RedProtocol and LSMProtocol simply extends REDLSMCommon defining the two
  methods previously described following the given specifications.
  
  \item[SimulationClient.] It is the real client that coordinates the different
  tests in a simulation (there are NSIM tests). It should be the only class accessed 
  by user to manage the simulation. It has an inner class 
  `SimulationController' which is a thread responsible for managing the 
  simulation and which coordinates the various pause and stop
  requests given from user.
  
  The possible pause which can occur can be: pause of one test, pause of one 
  simulation. The first consists on asking to the Hypervisor of a currently
  active simulation to pause every ProtocolThread executing, so every thread
  is put in not-runnable state. The latter consists on waiting for the 
  current test to terminate and pause every thread before the following test 
  starts (i.e. if the configuration file indicates to execute 4 tests with 
  NSIM=4 and the user asks to pause the simulation during the second test, the
  system terminates the second test and is paused before the third begins).
\end{description}

By the way other notable characteristics should be: the usage of abstract 
factories to build the graph (given the ray read from configuration file)
and to build the protocol in order to obtain an high extensibility, the 
implementation of the number read from satellite used in RED protocol
as a pseudorandom numbers generator, the usage of md5 cryptographic hash has 
part of the hashing function used on RED, the usage of event driven 
architecture to allow protocols to communicate with the GUI, etc...

\subsection{GUI}
The GUI part consists in set of panels build up to allow user to specify
parameters to run and manage the simulation. These panels are used as 
listeners connected to the threads to allow the GUI to know the state of
the simulation.

The GUI is also responsible of loading and parsing the configuration file.

\subsection{Server}
The Server part consists in a simple RMI server that accepts statistical data
collected in each simulation.

\section{Architectural choices}
The specification given to build the project left some decision to be taken by
developers in order to satisfy better the requests. In this section there are 
notable choices taken.
\begin{itemize}
  \item The parser has been developed as a mere syntactic controller. It just
  controls that the file asked to parse has the right grammar and saves in a 
  data structure every couple (key,value). This means that if in the file there
  is the line ``good morning = hello'' it is correctly parsed and saved in the
  parser data structure. The control on data types is done 
  by clients. This decision was taken because it has been thought that the 
  parser should be able to parse every correct file and if the application is 
  extended to protocols that need other parameters, it should always be able to
  parse their file.
  \item The simulation will stop as soon as a clone is detected. This means 
  that when a ProtocolThread realizes that it has a clone evidence, from that
  point on at least one more message is processed by the other threads until
  they reach a synchronization point. This is done to have the most precise
  approximation possible to determine the exact moment a clone attack is detected.
  \item During the location claim the cost to warn every neighbour of a 
  StationNode is the cost of sending just one message. This is done because in
  real life wireless communication the cost of perform a broadcast is the same
  as perform a send to a single station.
  \item StationNodes use mailboxes in order to obtain an asynchronous
  communication. The usage of mailboxes grant the possibility to have less 
  points to synchronize on and a simpler management of the different 
  behaviours because a ProtocolThread can determine what to do just inspecting 
  what kind of message it must process.
  
  Because of the mailbox oriented communication previously described, one
  drawback that exists in this architecture is the impossibility to 
  deterministically say whether or not a message will be processed. Since there
  are different threads that could access a StationNode mailbox, it is possible
  that even if in a certain moment that node is alive (i.e. has more than 0 
  battery) and a message is sent to it, 
  it won't be able to process that message because some other node put
  a message on its mailbox or because it is doing some energy sucking operation. 
  This means that some message, when a StationNode is low battery, can be sent 
  but never processed or even lost.
  
  Actually the behaviour of the system in managing dead StationNode is the
  following: if a StationNode is dead then it is not considered while 
  computing the nearest node in the routing algorithm, if it is alive and it
  is the nearest then it is used as destination of the forward message even if 
  meanwhile it died. The message in the latter case is lost.
  \item The RMI server must be specified only with the host name it has been
  loaded on. This is done because it should not be allowed for two servers
  to execute in the same machine because they could be configured to write 
  in the same file causing inconsistency.
\end{itemize}