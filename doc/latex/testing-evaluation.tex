\section{Testing and evaluation}
Thanks to the possibility of loading a local configuration file through the
local parser, it was possible to test a lot of configurations and the
results were in line with the expectations.

Parameters of the article \textit{M. Conti, R. Di Pietro, L. Mancini, and A. Mei,
``Distributed detection of clone attacks in wireless sensor networks'' IEEE 
Transactions on Dependable and Secure Computing, vol. 8, pp. 685/698,
September 2011} were also used to make some tests on the program and make a
comparison between the results reported on the paper and produced by the
application. This was done because it was interesting to determine if the 
statistical data collected by the developed application was comparable to
already present data known to be correct. 

In both RED and LSM protocols the two results were very close except for some
slight differences present in particular when a clone attack was detected.
After a meeting with professors Conti and Canlar it arose that differences may
be attributable to the fact that the simulator used in the article didn't stop
the simulation immediately after the clone was detected but only when every 
message in the nodes mailbox was extracted or every sensor died. This explains
the measured differences because the behaviour of the two simulators is 
different exactly in this point.

Other test were made to determine if the other aspects of the the application
were correctly developed.
\begin{itemize}
  \item The parser has been tested to verify that it refused malformed files or 
  files containing incorrect data. During tests, every malformed file and 
  every file with incomplete or wrong parameters were correctly refused. 
  As expected the parser didn't refused files which had more parameters in them
  than expected. Those parameters were simply ignored by client but still 
  present in the parser data structure.
  \item Server-client communication has been tested running server and client in
  different computers. We noticed a significant lag when the server were loaded
  in a GNU/Linux system while with other configuration everything worked 
  correctly. The cause of this behaviour wasn't determined because of lack of 
  time.
\end{itemize}

During protocols evaluation, it was observed that with almost all 
configurations RED protocol is far more effective than LSM. 
It uses a very small amount of energy and, with parameters n=300, p=0.1 and 
g=1, it detects clones almost every time, on the contrary of LSM that is more
expensive and, with the above parameters, ineffective to find clones 
(the probability for LSM to detect two clones was approximately 1/200).

An interesting fact that has been observed is that it's necessary to find a 
critical threshold of nodes for both protocols in order to make a clone 
detection possible. 
In fact it has been noticed that under 300 nodes there is an high probability 
that the graph is disconnected so, if clones are in two different connected
components, it is impossible to have a clone detection.

As last note we'd like to warn that package `test' and its subpackages should
not be considered while evaluating the application. They were used during 
development cycle to test the different software units and have not been 
documented. However it was kept because it contains file 
`FloodingProtocol.java' which models a naive example of possible clone
detection protocol. We though it was a conclusive proof of the extensibility 
of the program.